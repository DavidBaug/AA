\PassOptionsToPackage{unicode=true}{hyperref} % options for packages loaded elsewhere
\PassOptionsToPackage{hyphens}{url}
%
\documentclass[]{article}
\usepackage{lmodern}
\usepackage{amssymb,amsmath}
\usepackage{ifxetex,ifluatex}
\usepackage{fixltx2e} % provides \textsubscript
\ifnum 0\ifxetex 1\fi\ifluatex 1\fi=0 % if pdftex
  \usepackage[T1]{fontenc}
  \usepackage[utf8]{inputenc}
  \usepackage{textcomp} % provides euro and other symbols
\else % if luatex or xelatex
  \usepackage{unicode-math}
  \defaultfontfeatures{Ligatures=TeX,Scale=MatchLowercase}
\fi
% use upquote if available, for straight quotes in verbatim environments
\IfFileExists{upquote.sty}{\usepackage{upquote}}{}
% use microtype if available
\IfFileExists{microtype.sty}{%
\usepackage[]{microtype}
\UseMicrotypeSet[protrusion]{basicmath} % disable protrusion for tt fonts
}{}
\IfFileExists{parskip.sty}{%
\usepackage{parskip}
}{% else
\setlength{\parindent}{0pt}
\setlength{\parskip}{6pt plus 2pt minus 1pt}
}
\usepackage{hyperref}
\hypersetup{
            pdfborder={0 0 0},
            breaklinks=true}
\urlstyle{same}  % don't use monospace font for urls
\setlength{\emergencystretch}{3em}  % prevent overfull lines
\providecommand{\tightlist}{%
  \setlength{\itemsep}{0pt}\setlength{\parskip}{0pt}}
\setcounter{secnumdepth}{0}
% Redefines (sub)paragraphs to behave more like sections
\ifx\paragraph\undefined\else
\let\oldparagraph\paragraph
\renewcommand{\paragraph}[1]{\oldparagraph{#1}\mbox{}}
\fi
\ifx\subparagraph\undefined\else
\let\oldsubparagraph\subparagraph
\renewcommand{\subparagraph}[1]{\oldsubparagraph{#1}\mbox{}}
\fi

% set default figure placement to htbp
\makeatletter
\def\fps@figure{htbp}
\makeatother


\date{}

\begin{document}

\hypertarget{header-n0}{%
\subsection{Aprendizaje Automático - Cuestionario 1}\label{header-n0}}

\hypertarget{header-n2}{%
\subparagraph{David Gil Bautista}\label{header-n2}}

\hypertarget{header-n3}{%
\subsubsection{Preguntas}\label{header-n3}}

\textbf{1) Identificar, para cada una de las siguientes tareas, que tipo
de aprendizaje es el adecuado (supervisado, no supervisado, por
refuerzo) así como los datos de aprendizaje que deberiamos usar en su
caso. Si una tarea se ajusta a más de un tipo, explicar como y describir
los datos para cada tipo.}

\begin{verbatim}
a) Dada una colección de fotos de caras de personas de distintas razas establecer cuantas razas distintas hay representadas en la colección.
\end{verbatim}

Para este problema tenemos una serie de datos y debemos categorizarlos
en distintos tipos utilizando sus características. Se trata de un
problema de aprendizaje no supervisado, ya que, a partir de unos datos
de entrada establecemos una serie de relaciones entre ellos para
determinar el número de razas distintas que tenemos. Se podría tener un
vector de características (vector de flotantes) en las que se
presentaran los distintos rasgos; color de ojos, anchura de nariz,
grosor de labios... etc. Las mismas razas tendrán valores similares y al
representar los datos veremos que cada raza se agrupará en un espacio.

\begin{verbatim}
b) Clasificación automática de cartas por distrito postal
\end{verbatim}

Si tratamos este problema como uno de predicción podríamos usar
aprendizaje supervisado y a partir de una serie de datos de
entrenamiento calcular una función que nos permita clasificar las nuevas
cartas. Para este caso necesitaríamos una serie de datos para calcular
nuestra función, las cartas con sus distritos postales y su etiqueta. Es
cierto, también, que podríamos tratar este problema con aprendizaje no
supervisado, ya que, como en el apartado anterior, podríamos representar
los datos para ver como se estructuran al representarlos y ver las
relaciones que hay entre ellos.

\begin{verbatim}
c) Decidir si un determinado índice del mercado de valores subirá o bajará dentro de un periodo de tiempo fijado.
\end{verbatim}

Este problema podríamos tratarlo aplicando la teoría de juegos y el
aprendizaje por refuerzo. Para estimar si un índice del mercado de
valores subirá o bajará deberíamos estudiar todas las posibilidades que
tiene y elegir la más probable.

\begin{verbatim}
d) Aprender un algoritmo que permita a un robot rodear un obstaculo.
\end{verbatim}

Aprendizaje por refuerzo. Para este problema supondré que el robot tiene
una serie de sensores y memoria. Sabiendo esto se me ocurriría generar
un árbol de decisión que se generará mediante los datos de entrada del
robot y el objetivo (que es rodear el objeto). Al ver un obstáculo el
robot solo tomará una serie de decisiones (secuencia de pasos), lo que
equivale a aprender un algoritmo.

\textbf{2) ¿Cuales de los siguientes problemas son más adecuados para
una aproximación por aprendizaje y cuales más adecuados para una
aproximación por diseño? Justificar la decisión.}

\begin{verbatim}
a)Agrupar los animales vertebrados en mamíferos, reptiles, aves, anfibios y peces.
\end{verbatim}

Para agrupar un grupo de datos en distintas categorías usaría una
aproximación por aprendizaje, ya que, a partir de una sería de datos
iniciales con sus etiquetas calcularía una función que me permitiera
predecir a que familia de animales pertenecen una serie de datos y
agruparlos.

\begin{verbatim}
b)Determinar si se debe aplicar una campaña de vacunación contra una enfermedad.
\end{verbatim}

Para este problema sería mejor aplicar una aproximación por diseño.
Digamos que tenemos una enfermedad y una serie de datos probabilísticos
asociados a dicha enfermedad que se recogen cada año. Con esos datos
podríamos construir una distribución de probabilidad que nos permita
secidir si se aplica una campaña de vacunación o no.

\begin{verbatim}
c) Determinar si un correo electrónico es de propaganda o no.
\end{verbatim}

Debemos predecir si un correo electrónico es SPAM o no. Podríamos
obtener una función a partir de un conjunto de datos de entrenamiento y
sus etiquetas que leyera un vector con las palabras y a partir de este
determinara si es SPAM o no.

\begin{verbatim}
d) Determinar el estado de ánimo de una persona a partir de una foto de su cara.
\end{verbatim}

En este caso contamos con una serie de datos iniciales, un vector con
las facciones faciales, y sus etiquetas. A partir de esos datos usando
aproximación por aprendizaje tendríamos una función que nos permita
determinar el estado de ánimo de una persona a partir de esos datos.

\begin{verbatim}
e) Determinar el ciclo óptimo para las luces de los semáforos en un cruce con mucho tráfico.
\end{verbatim}

Al no tratarse de de un problema mediante el cual tengamos que estimar
una salida a partir de unos datos iniciales podemos deducir que lo mejor
sería aplicar una aproximación por diseño.

\textbf{3) Construir un problema de aprendizaje desde datos para un
problema de clasificación de fruta en una explotación agraria que
produce mangos, papayas y guayabas. Identificar los siguientes elementos
formales \(X , Y, D, f\) del problema. Dar una descripción de los mismos
que pueda ser usada por un computador. ¿Considera que en este problema
estamos ante un caso de etiquetas con ruido o sin ruido? Justificar las
respuestas.}

Nuestros elementos formales serán los siguientes:

X : Vector de flotantes en los que representamos las siguientes
características: Carbohidratos, Grasas, Proteínas y Agua. Todos estos
datos medidos por cada 100 gramos de la pieza.

Y : Nuestra etiqueta sería el tipo de fruta \{mango, papaya, guayaba\}
identificados por una etiqueta única.

D : Conjunto de datos de entrada para entrenar.

f : Función tal que dado un vector con esos 4 elementos determine a que
tipo se ajusta mejor un dato.

Nos encontramos ante un problema con etiquetas con ruido puesto que para
una determinada pieza de fruta puede que sus características determinen
que se trate de una distinta a la que es.

\textbf{4) Sea X una matriz de números reales de dimensiones
\(N × d, N > d.\) Sea \(X = UDV^T\) su descomposición en valores
singulares (SVD). Calcular la SVD de \(X^TX\) y \(XX^T\) en función de
la SVD de \(X\). Identifique dos propiedades de estás nuevas matrices
que no tiene \(X\). ¿Qué valor representa la suma de la diagonal
principal de cada una de las matrices producto?}

\[X^TX = V D U^TU D V^T = VD^2V^T\\
XX^T=U D V^TV D U^T = UD^2U^T\]

 Cuando tenemos un producto de matrices y necesitamos calcular su
inversa es el mismo producto en orden invertido en el que cada matriz es
su traspuesta. De esta forma obtenemos \(X^T\).

Al hacer \(X^TX\) tenemos \(V D U^TU D V^T\) donde podemos aplicar una
propiedad de las matrices ortogonales en la que al multiplicar una
matriz por su traspuesta obtenemos la matriz identidad. Obviando esta
matriz, puesto que su producto por cualquier otra genera esa misma
matriz, obtenemos \(VD^2V^T\).

1) La matriz U es una matriz ortogonal y la matriz D es una matriz
diagonal.

2) El orden de la multiplicación supone que obtengamos la descomposición
en valores singulares en función de U o de V.

3) Sea \((XX^T)^n\) con \(n >= 1\) tenemos que su descomposición en
valores singulares es \(U D^N U^T\).

4) Sea \((X^TX)^n\) con \(n >= 1\) tenemos que su descomposición en
valores singulares es \(V D^N V^T\)

5) Las matrices \(X^TX\) y \(XX^T\) son simétricas.

\textbf{5) Sean x e y dos vectores de características de dimensión
\(M × 1\). La expresión}

\[cov(x,y) = \frac{1}{M}  \sum_{i=1}^{M}(x_{i} - \bar{x})(y_{i} - \bar{y})\]

\textbf{define la covarianza a entre dichos vectores, donde \(\bar{z}\)
representa el valor medio de los elementos de z. Considere ahora una
matriz X cuyas columnas representan vectores de características. La
matriz de covarianzas asociada a la matriz
\(X = (x_1, x_2, · · · , x_N )\) es el conjunto de covarianzas definidas
por cada dos de sus vectores columnas. Es decir}

\[cov(X) = \begin{pmatrix}
cov(x_1,x_1)&cov(x_1,x_2)&... &cov(x_1,x_N)\\
cov(x_2,x_1)&cov(x_2,x_2)&...&cov(x_2,x_N)\\
...&...&...&...\\
cov(x_N,x_1)&cov(x_N,x_2)&...&cov(x_N,x_N)
\end{pmatrix}\]

\textbf{Sea \(1_{M}^T = (1,1,...,1)\) un vector \(M \times 1\) de unos.
Mostrar que representan las siguientes expresiones.}

\begin{enumerate}
\def\labelenumi{\arabic{enumi}.}
\item
  \(E1 = 11^T X\)

  Si tenemos que \(1^T\) es un vector de M unos, al multiplicar por 1
  obtenemos una matriz \(1_{MxM}\), es decir, una matriz cuadrada de
  orden M rellena de unos.

  Al multiplicar dicha matriz por X tenemos un producto matricial en el
  que sumamos todas las columnas de X, es decir,
  tendremos el sumatorio de la columna X,

  Sabiendo que E1 es la sumatoria de las columnas, al dividir entre M
  tenemos una matriz en la que tenemos la media de las columnas de X.

  Al multiplicar \((X - \bar{X})^T\) por \((X - \bar{X})\) obtenemos una
  matriz con la sumatoria del producto de las columnas, lo que podemos
  representar como \(\sum_{i=1}^{M} (x_i-\bar{x})(y_i-\bar{y})\).

  De este modo hemos demostrado que
  \(E2 = \sum_{i=1}^{M} (x_i-\bar{x})(y_i-\bar{y}) = M\;cov(x,y)\)
\end{enumerate}

6) Considerar la matriz \emph{hat} definida en regresión,
\(H = X(X^TX)^{−1}X^T\), donde X es una matriz

\(N × (d + 1)\), y \(X^TX\) \textbf{es invertible.}

\emph{a) Mostrar que H es simétrica.}

 Si H es simétrica podemos decir que \(H = H^T\)

\[H = H^T\\
X(X^TX)^{-1}X^T = (X(X^TX)^{-1}X^T)^T\\
X(X^TX)^{-1}X^T = X ((X^TX)^{-1})^TX^T\\
X(X^TX)^{-1}X^T = X ((X^TX)^T)^{-1}X^T\\
X(X^TX)^{-1}X^T = X(X^TX)^{-1}X^T\\
H = H^T \\\]

 Según las propiedades de las matrices al hacer la traspuesta de un
producto de matrices se colocan en orden inverso y haciendo su
traspuesta mediante lo que obtenemos \(X ((X^TX)^{-1})^TX^T\\\). Una vez
hecho esto intercambiamos el orden de las potencias en
\(((X^TX)^{-1})^T\) para obtener \(((X^TX)^T)^{-1}\) y aplicando la
misma propiedad del producto de traspuestas obtenemos que \((X^TX)^T\)
es \((X^TX\). Al final obtenemos que \(H^T = X(X^TX)^{-1}X^T\) por lo
que demostramos que H es simétrica.

\emph{b) Mostrar que es idempotente \(H^2 = H\)}

\[H^1 = H\\
H^2 = H\\
H^2 = H H= H\\
H = X(X^TX)^{-1}X^T \\
HH =X(X^TX)^{-1}X^TX(X^TX)^{-1}X^T\\
HH =X(X^TX)^{-1}X^T\]

 Al hacer \(H^2\) podemos ver que obtenemos \(X^TX(X^TX)^{-1}\) y según
las propiedadesde las matrices; una matriz por su inversa es la
identidad. Haciendo esto podemos ver que obtenemos la matriz original.
De este modo demostramos que H es una matriz idempotente.

\emph{c) ¿Qué representa la matriz H en un modelo de regresión?}

 La matriz H es la matriz de derivadas de segundo orden que indica el
avance en la dirección del gradiente.

\textbf{7) La regla de adaptación de los pesos del Perceptron
\((w_{new} = w_{old} + yx)\) tiene la interesante propiedad de que los
mueve en la dirección adecuada para clasificar x de forma correcta.
Suponga el vector de pesos \emph{w} de un modelo y un dato \(x(t)\) mal
clasificado respecto de dicho modelo. Probar que la regla de adaptación
de pesos siempre produce un movimiento en la dirección correcta para
clasificar bien \(x(t)\).}

 Tenemos que demostrar que si en una iteración hemos clasificado mal un
dato en la siguiente iteración el perceptron se moverá de forma que lo
clasifique bien. Para ello tenemos lo siguiente:

\[y(t)(w^T(t)x(t))  < 0\\
y(t)(w^T(t+1)x(t))  > 0\\
y(t)w^T(t+1)x(t)>y(t)w^T(t)x(t)\\
y(t)(w(t)+y(t)x(t))^Tx(t)>y(t)w^T(t)x(t)\\
y(t)w^T(t)+1x^T(t)x(t)>y(t)w^T(t)x(t)\\
y(t)w^T(t)+x^2(t)>y(t)w^T(t)x(t)\]

 Si está mal clasificado la y será distinta de \(w^Tx\), por lo que al
ser uno negativo siempre será menot que 0. Si está bien clasificado
puede que ambos sean negativos o ambos positivos, por lo que la
multiplicación siempre tendrá signo positivo. Si
\(y(t)(w^T(t)x(t))  < 0\), está mal clasificado, en la siguiente
iteración \(y(t)(w^T(t+1)x(t))  > 0\), estará bien clasificado, por lo
que podemos decir que \(y(t)w^T(t+1)x(t)>y(t)w^T(t)x(t)\).

 Desarrollando esta expresión y sabiendo que \(w_{new} = w_{old}+yx\)
tenemos que \(y(t)(w^T(t)+y(t)x(t))x(t)>y(t)w^T(t)x(t)\). Al multiplicar
la etiqueta por la etiqueta vamos a obtener un 1 multiplicado por la
traspuesta de x, y al multiplicar dicha traspuesta por x obtenemos
\(x^2\) lo que garantiza que vaya a ser positivo y por tanto mayor que
en la iteración anterior.

\textbf{8) Sea un problema probabilistico de clasificación binaria cuyas
etiquetas son \(({0,1})\), es decir \(P(Y = 1) = h(x)\) y
\(P(Y = 0) = 1 − h(x)\)}

\emph{a) Dar una expresión para \(P(Y)\) que sea válida tanto para
\(Y=1\) como para \(Y=0\)}

\emph{b) Considere una muestra de \(N\) v.a. independientes. Escribir la
función de Máxima Verosimilitud para dicha muestra.}

\emph{**c) Mostrar que la función \(h\) que maximiza la verosimilitud de
la muestra es la misma que minimiza}

\[E_{in}(w) = 	\sum_{n=1}^{N} || y_n = 1|| ln \frac{1}{h(x_n)}+||y_n=0||ln \frac{1}{1-h(x_n)}\]

\emph{donde \(||\cdot||\) vale 1 o 0 según que sea verdad o falso
respectivamente la expresión en su interior.}

\emph{d) Para el caso \(h(x) = σ(w^T x)\) mostrar que minimizar el error
de la muestra en el apartado anterior es equivalente a minimizar el
error muestral}

\[E_{in}(w) = 	\frac{1}{N}\sum_{n=1}^{N} ln (1 + e^{-y_nw^Tx_n})\]

\textbf{9) Mostrar que en regresión logística se verifica:}

\[\nabla E_{in}(w) = - \frac{1}{N} \sum_{n=1}^{N} \frac{y_nx_n}{1+e^{y_nw^Tx_n}} = \frac{1}{N} \sum_{n=1}^{N}-y_nx_nσ( -y_nw^Tx_n)\]

Argumentar sobre si un ejemplo mal clasificado contribuye al gradiente
más que un ejemplo bien clasificado.

\[\sigma(x) = \frac{1}{1+e^{-x}}=\frac{e^x}{e^x+1}\\
\nabla E_{in}(w) = - \frac{1}{N} \sum_{n=1}^{N} \frac{y_nx_n}{1+e^{y_nw^Tx_n}} = \frac{1}{N} \sum_{n=1}^{N}-y_nx_nσ( -y_nw^Tx_n)\\
\nabla E_{in}(w) = - \frac{1}{N} \sum_{n=1}^{N} \frac{y_nx_n}{1+e^{y_nw^Tx_n}} = \nabla E_{in}(w) = - \frac{1}{N} \sum_{n=1}^{N} y_nx_n\frac{1}{1+e^{y_nw^Tx_n}}\\
\sigma(-y_nw^Tx_n) = \frac{1}{1+e^{y_nw^Tx_n}}\\
-\frac{1}{N}\sum_{n=1}^{N} y_nx_n\sigma(y_nw^Tx_n)\\
\frac{1}{N}\sum_{n=1}^{N} -y_nx_n\sigma(-y_nw^Tx_n) =  - \frac{1}{N} \sum_{n=1}^{N} \frac{y_nx_n}{1+e^{y_nw^Tx_n}}\]

Despejando la fracción obtenemos \(\frac{1}{1+e^{y_nw^Tx_n}}\) y como ya
hemos visto en la expresión de sigma, esto equivale a
\(\sigma(y_nw^Tx_n)\). Con esto ya podemos demostrar que ambas
expresiones son equivalentes.

Cuando tenemos un ejemplo mal clasificado, la potencia de
\(e^{y_nw^Tx_n}\) será negativa puesto que al ser de distinto signo la
etiqueta del vector de pesos por el de características el signo será
negativo. Cuando la potencia de e es un número negativo, e tiende a ser
un número pequeño mientras que si la potencia es positiva, e tiende a
crecer. Si la potencia de e es un número negativo, la fracción
\(\frac{1}{1+e^{y_nw^Tx_n}}\) será menor que si la potencia de e es un
número positivo.

De esta forma demostramos que si un ejemplo está mal clasificado su
error será más significativo y por tanto más contribuyente que si está
bien clasificado.

\textbf{10) Definamos el error en un punto \((x_n, y_n)\) por:}

\[e_n(w) = max(0, -y_nw^Tx_n)\]

Argumentar si con esta función de error el algoritmo PLA puede
interpretarse como SGD sobre \(e_n\) con tasa de aprendizaje \(ν = 1\).

\end{document}
